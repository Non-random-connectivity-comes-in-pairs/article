
We were able to show that a relative overrepresentation of bidirectional connections arrises necessarily in networks with a non-degenerate distribution of symmetric connection probabilities. Conversely, an absence of an overabundance of reciprocal pairs, as for example found in the intra-layer connectivity of the mouse C2 barrel column \cite{Lefort2009}, points towards either a truly random network or an asymmetry in the connection probabilities. 


Quantitatively, a network in which connection probabilities take on one of two values is easily able to account for even the highest values of overrepresentation reported. A network with such a two-point distribution of connection probabilities can be understood as two populations 

th connection probability distributions analyzed, the two-point distribution and the gamma distribution, allowed for configurations in which high values of $\varrho$ can be obtained. 


We find that in networks with symmetric connection probabilities, an overrepresentation of 

* Summary of the results: Necessary overrepresentation in networks with symmetric, non-degenerate distribution of connection probabilities.

* Summary of the results: Two point distribution is highly relevant (\cite{Lee2016}) and is able to produce overrepresentation as found by \cite{Song2005}.

* Summary of the results: To achieve high values of $\varrho$, continuous distribution (Gamma) needs to have a \enquote{long tail} and 

* What are mechanisms that could affect connection probabilities in a symmetric manner? $\rightarrow$ Distance-dependency (Song vs. Perin), ...

* What are mechanisms that favor a certain direction of connection? $\rightarrow$ Morphology, ...

* Conclusion: Overrepresentation of bidirectional emerges as a \enquote{symptom} of \textbf{any} non-random connectivity $+$ symmetric connectivity, thus not necessarily an intrinsic parameter of network connectivity

* Research question should thus possibly focus more on 1) higher order non-random structures or 2) weights of connectivity (example: bidirectionally connected pairs typically have strong weights). 

* Absence of overrepresentation of bidirectional does not necessarily allow conclusion that network is completely random, might be asymmetric in connection probabilities. However, an absence of overrepresentation could can be hint towards either a close to random network or network that has strong asymmetry.


* Why this is an important result: Overrepresentation of bidirectional connections maybe more a symptom of non-randomness + symmetry. Insight allows the formulation of possibly more important questions (weights of bidirectional connections etc..)
