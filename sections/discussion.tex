

* Summary of the results: Necessary overrepresentation in networks with symmetric, non-degenerate distribution of connection probabilities.

* Summary of the results: Two point distribution is highly relevant (\cite{Lee2016}) and is able to produce overrepresentation as found by \cite{Song2005}.

* Summary of the results: To achieve high values of $\varrho$, continuous distribution (Gamma) needs to have a \enquote{long tail} and 

* What are mechanisms that could affect connection probabilities in a symmetric manner? $\rightarrow$ Distance-dependency (Song vs. Perin), ...

* What are mechanisms that favor a certain direction of connection? $\rightarrow$ Morphology, ...

* Conclusion: Overrepresentation of bidirectional emerges as a \enquote{symptom} of \textbf{any} non-random connectivity $+$ symmetric connectivity, thus not necessarily an intrinsic parameter of network connectivity

* Research question should thus possibly focus more on 1) higher order non-random structures or 2) weights of connectivity (example: bidirectionally connected pairs typically have strong weights). 

* Absence of overrepresentation of bidirectional does not necessarily allow conclusion that network is completely random, might be asymmetric in connection probabilities. However, an absence of overrepresentation could can be hint towards either a close to random network or network that has strong asymmetry.

* Absence of bidirectional connection overrepresentation was found in barrel cortex by \cite{Lefort2009}. As simultaneously the network was reported to have strong non-random connectivity, we conclude that that strong directionality was present in the network (L4 mouse barrel cortex). (What are the possible reasons/consequences of this? Should read the paper more carefully.)

\blockquote{Lefort 2009: \enquote{Altogether within L4, we found 254
    excitatory synaptic connections among 1046 tested connections,
    giving an average probability of any two excitatory L4 neurons
    being synaptically connected as 24.3\% (which we will denote as
    PL4/L4 = 24.3\%). The peak L4/L4 uEPSP amplitudes ranged from 0.06
    mV to 7.79 mV (mean ± SEM = 0.95 ± 0.08 mV; median = 0.52 mV)
    (Table 2). We also quantified the uEPSP kinetics (see Table S1 and
    Figure S1 available online). \cc{Among these synapti- cally connected
    pairs of neurons, we found that 59 were recipro- cally
    bidirectionally connected. This is close to the value of 62
    reciprocal connections expected for a randomly wired network
    (given by 24.3\% 3 24.3\% 3 1046) (Table S2). However, in other
    respects the neocortical excitatory synaptic pathways appear far
    from randomly organized. For example, we found evidence for highly
    specific patterns of connectivity between L4 and the other
    layers.} Excitatory output from L4 to all other cortical layers is
    substantial, with over 10\% connectivity for L4/L2, L4/L3, and
    L4/L5A (PL4/L2 = 12\%, PL4/L3 = 14.5\%, and PL4/L5A = 11.6\%). The
    mean output connectivity from L4 to other layers is 10.8\%. In
    contrast, it receives very little input from the other cortical
    layers, with the strongest input to L4 originating from L3 with
    PL3/L4 = 2.4\% and the mean interlaminar input connec- tivity to
    L4 being 1.0\%}}
