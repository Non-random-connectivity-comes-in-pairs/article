
%% Understanding the wiring of local cortical circuits is an open problem,  Both experiemental and theoretical evidence points that networks,

Experimental evidence suggests that any pair of excitatory cells within a cortical column has contact points between axon and dendrite close enough to support a synaptic connection between the cells \cite{Stepanyants2004,Kalisman2005}.
%
Despite this potential \enquote{all-to-all} connectivity, only a small fraction of the contacts are realized as functional synapses.
%
Uncovering the underlying principles of which contact points get utilized for synaptic transmission is crucial for our understanding of the structure and function of the local cortical circuits in the mammalian brain.


The emerging local networks in the rat visual and somatosensory cortex have been shown to feature non-random structure \cite{Song2005, Perin2011} and much attention was given to bidirectionally connected neuron pairs that are occurring more often than expected from random connectivity \cite{Bourjaily2011, Clopath2010, Miner2016}.
%
In this study we have shown a condition under which non-random network structure and the occurrence of reciprocally connected pairs are inherently linked; a relative overrepresentation of bidirectional connections arises necessarily in networks with a non-degenerate distribution of symmetric connection probabilities.
%
Absence of an overabundance of reciprocal pairs on the other hand, as for example found in the intra-layer connectivity of the mouse C2 barrel column \cite{Lefort2009}, points towards either a truly random network or an asymmetry in the connection probabilities. 


Quantitatively, a network in which connection probabilities take on one of two values is easily able to account for even the highest values of overrepresentation reported.
%
A network with such a two-point distribution of connection probabilities might occur naturally, where the probability of connection depends on whether a given pair of neurons shares a certain feature, for example has a similar orientation preference or not \cite{Lee2016a}.


A continuous distribution in connection probabilities on the other hand might occur when pair connectivity depends on a continuous parameter such as the inter-neuron distance or the neurons' age.
%
We showed that networks in which connection probabilities follow a gamma distribution can as well have a high relative occurrence of reciprocally connected pairs, however in this case a larger fraction of pairs remain unconnected with a very high probability.


It is likely that a combination of such effects determines the connection probabilities in local cortical networks.
%
Importantly, we showed that as long as this probability is symmetric for pairs, any such effect that creates a non-degenerate distribution of probabilities will cause an increase of the reciprocity in the network.

%% When an overrepresentation of reciprocal pairs is absent in a cortical circuit, could one argue that neither distance-dependent nor feature specific connectivity is present in the network?
%% %
%% If there is no evidence for a substantial asymmetry in the network's connectivity, we would in fact conclude from the absence of an elevated relative occurrence of bidirectional connections that any effect on the pair connection probabilities is either very small or that multiple effects cancel out such that connectivity is well represented by an Erd\H{o}s-R\'{e}nyi random graph model.
%
%% Following this logic, we predict from the absence of an overrepresentation of bidirectional connections determined in the experiments \textcite{Lefort2009}, that there is no prevalent feature in the C2 mouse barrel cortex that by itself can predict connection probabilities of intra-layer circuits, as we would in this case, assuming symmetric connection probabilities, expect to see an overrepresentation of bidirectional connections. 

Our results confirm the intuitive notion that reciprocity is favored in symmetric networks, whereas asymmetric probabilities of connection inhibit the occurrence of bidirectionally connected pairs. Network models with symmetric connectivity such as Hopfield nets generally excel at memory storage and retrieval through fixed point attractor dynamics \cite{Hopfield1982}, while asymmetric network models such as synfire chains are suitable for reliable signal transmission \cite{Abeles1982, Diesmann1999}. This suggests the intriguing possibility that one may be able to infer the nature of the computations in a neural circuit based on certain statistics of its connectivity such as the abundance of bidirectionally connected pairs.

In conclusion, the present study puts the overrepresentation of bidirectional connections found in local cortical circuits in a new light.
%
If connection probabilities are symmetric in pairs, the overrepresentation emerges as a symptom of any form of non-random connectivity.
%
It is thus crucial for both future experimental and modeling studies to develop a more refined view of non-random network connectivity that goes beyond simple pair statistics. Focusing on higher order connectivity patterns and taking into account the actual synaptic efficacies seem promising avenues for future research into the non-random wiring of brain circuits.
