
We have shown that a relative overrepresentation of bidirectional connections arises necessarily in networks with a non-degenerate distribution of symmetric connection probabilities. Conversely, an absence of an overabundance of reciprocal pairs, as for example found in the intra-layer connectivity of the mouse C2 barrel column \cite{Lefort2009}, points towards either a truly random network or an asymmetry in the connection probabilities. 

Quantitatively, a network in which connection probabilities take on one of two values is easily able to account for even the highest values of overrepresentation reported. A network with such a two-point distribution of connection probabilities is likely two occur naturally, where the probability of connection depends on whether a given pair of neurons shares a certain feature, such as a common orientation preference \cite{Lee2016}, or not. 

A continuous distribution in connection probabilities on the other hand might occur when pair connectivity depends on a continuous parameter such as the inter-neuron distance or the neurons' age. We showed that networks in which connection probabilities follow a gamma distribution can as well have a high relative occurrence of reciprocally connected pairs, however in this case a larger fraction of pairs remains almost certainly unconnected.

It is likely that a combination of both the discrete and continuous effects determines the connection probabilities in local cortical networks. Importantly, we showed that as long as this probability is symmetric for pairs, any such effect that creates a non-degenerate distribution of probabilities will cause an increase of the reciprocity in the network.

When an overrepresentation of reciprocal pairs is absent in a cortical circuit, could one argue that neither distance-dependent nor feature specific connectivity is present in the network? If there is no evidence for a substantial asymmetry in the network's connectivity, we would in fact conclude from the absence of an elevated relative occurrence of bidirectional connections that any effect on the pair connection probabilities is either very small or that multiple effects cancel out to give an overall connection probability that accurately predicts connectivity across all pairs. \cc{[Felix: I added a litte bit of information in the next sentence. But I see your concern. Let's talk about this first before I spend a lot of time rewriting something we're not going to keep in the article]} Following this logic, we predict from the absence of an overrepresentation of bidirectional connections determined in the experiments \textcite{Lefort2009}, that there is no prevalent feature in the C2 mouse barrel cortex that by itself can predict connection probabilities of intra-layer circuits, as we would in this case, assuming symmetric connection probabilities, expect to see an overrepresentation of bidirectional connections. 

The present study puts the overrepresentations bidirectional connections found in local cortical circuits in a new light. If connection probabilities are symmetric in pairs, the overrepresentation emerges as a symptom of any form of non-random connectivity. It is thus crucial for both future experimental and modeling studies to develop a more refined view of non-random network connectivity that goes beyond pair statistics and focuses on higher order connectivity or the distributions in synaptic weight in order to obtain a better understanding of the presence and importance of non-random wiring in brain networks.

%% This is an important insight that should inform future studies. Rather than focusing , it is important to develop a more refined view of non-random network connectivity, such as high-order motifs and the distribution synaptic weights. 




%% * Conclusion: Overrepresentation of bidirectional emerges as a \enquote{symptom} of \textbf{any} non-random connectivity $+$ symmetric connectivity, thus not necessarily an intrinsic parameter of network connectivity

%% * Research question should thus possibly focus more on 1) higher order non-random structures or 2) weights of connectivity (example: bidirectionally connected pairs typically have strong weights). 


%% * Why this is an important result: Overrepresentation of bidirectional connections maybe more a symptom of non-randomness + symmetry. Insight allows the formulation of possibly more important questions (weights of bidirectional connections etc..)



%% * What are mechanisms that could affect connection probabilities in a symmetric manner? $\rightarrow$ Distance-dependency (Song vs. Perin), ...

%% * What are mechanisms that favor a certain direction of connection? $\rightarrow$ Morphology, ...
