
We were able to show that a relative overrepresentation of bidirectional connections arrises necessarily in networks with a non-degenerate distribution of symmetric connection probabilities. Conversely, an absence of an overabundance of reciprocal pairs, as for example found in the intra-layer connectivity of the mouse C2 barrel column \cite{Lefort2009}, points towards either a truly random network or an asymmetry in the connection probabilities. 

Quantitatively, a network in which connection probabilities take on one of two values is easily able to account for even the highest values of overrepresentation reported. A network with such a two-point distribution of connection probabilities is likely two occur naturally, where the connection probability depends on whether a given pair of neurons shares a certain feature, such as a common orientation preference \cite{Lee2016}, or not. 

A continuous distribution in connection probabilities on the other hand might occur when pair connectivity depends on a continuous parameter such as the distance between neurons, the neurons' age or intrinsic parameters of connectivity. We showed that networks in which connection probabilities follow a gamma distribution can as well have a high relative occurrence of reciprocally connected pairs, however in this case a larger fraction of pairs remains almost certainly unconnected.

It is likely that a combination of both the discrete and continuous effects determines the connection probabilities in local cortical networks. We showed that as long as this probability is symmetric for pairs, any effect on .

When reciprocal pairs are absent, One could argue that neither distance-dependency nor functionally specific . As we have no reason to believe, might in fact be so small that no . Following this logic, we predict that no prevalent feature that by itself can predict as we would in this case, assuming symmetric connection probabilities, expect to see an overrepresentation of bidirectional connections.


%% * What are mechanisms that could affect connection probabilities in a symmetric manner? $\rightarrow$ Distance-dependency (Song vs. Perin), ...

%% * What are mechanisms that favor a certain direction of connection? $\rightarrow$ Morphology, ...

* Conclusion: Overrepresentation of bidirectional emerges as a \enquote{symptom} of \textbf{any} non-random connectivity $+$ symmetric connectivity, thus not necessarily an intrinsic parameter of network connectivity

* Research question should thus possibly focus more on 1) higher order non-random structures or 2) weights of connectivity (example: bidirectionally connected pairs typically have strong weights). 


* Why this is an important result: Overrepresentation of bidirectional connections maybe more a symptom of non-randomness + symmetry. Insight allows the formulation of possibly more important questions (weights of bidirectional connections etc..)
