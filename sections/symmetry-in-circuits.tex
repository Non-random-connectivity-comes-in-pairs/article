
-- How likely is the symmetry assumption in cortical circuits? --

However, even when pairs . This can be demonstrated by considering a model of in which .

Consider for this that connection probabilities $P_{ij}$ are distributed according to some probability density function $f_{P_{ij}}$. The expected probability of a reciprocal connection within a pair can then be expressed as
%
\begin{align}
  \E(P_{ij}P_{ji}) = \int_0^1 \int_0^1 xy\, f_{P_{ij},P_{ji}}(x,y) \, dx\, dy,
\end{align}
%
where $f_{P_{ij},P_{ji}}(x,y)$ is the joint probability density function of $P_{ij}$ and $P_{ji}$, 
%
\begin{align}
  f_{P_{ij},P_{ji}}(x,y) =  f_{P_{ji} | P_{ij}}(y \mid x) f_{P_{ij}}(x).
\end{align}
%
In the case that $P_{ij}$ and $P_{ji}$ are independent we have $f_{P_{ji} | P_{ij}}(y \mid x) = f_{P_{ji}}(y)$ and in the case of $P_{ij}=P_{ji}$ it is $f_{P_{ji} | P_{ij}}(y \mid x) = \delta(y-x)$.  Here we propose a model for the conditional density function that transitions between the two extreme cases by multiplying $f_{P_{ji}}(y)$ with the density function of a normal distribution centered around $x$,
%
\begin{align}
  f_{P_{ji} | P_{ij}} (y \mid x) = \frac{1}{N_{\sigma}(x)} f_{P_{ji}}(y)\, \frac{1}{\sigma \sqrt{2 \pi}} \,e^{\frac{(y-x)^2}{2 \sigma^2}} \label{eq:fpijpji}
\end{align}
%
The additional factor $N_{\sigma}(x)$  makes sure that $f_{P_{ji}|P_{ij}} (y \mid x)$ integrates to one,
%
\begin{align}
  N_{\sigma}(x) = \int_0^1 f_{P_{ji}}(z)\, \frac{1}{\sigma \sqrt{2 \pi}}\, e^{\frac{(z-x)^2}{2 \sigma^2}} \,dz.
\end{align}
%
Indeed as the variance $\sigma^2$ of the normal distribution increases $f_{P_{ji}|P_{ij}} (y \mid x) \to f_{P_{ji}}$ and in the limit $\sigma \to 0$ we have
\begin{align}
  \lim_{\sigma \to 0}   f_{P_{ji}|P_{ij}} (y \mid x) = \delta(y-x).
\end{align}
%
In Figure~\ref{fig:sym_fail}A the conditional density function is shown . Finally we can test
%
\begin{align}
  \varrho = \frac{\E(P_{ij} P_{ji})}{\mu^2},
\end{align}
%
where the overall connection probability $\mu$ is calculated as
%
\begin{align}
 \mu = \frac{1}{2} \int_0^1 x f_{P_{ij}}(x)\,dx + \frac{1}{2} \int_0^1 f_{P_{ij}}(x) \int_0^1 y \,f_{P_{ji}\vert P_{ij}}(y \mid x) \,dy \, dx.
\end{align}
%
The

Figure~\ref{fig:sym_fail}B

\begin{figure}[h!]
\centering
%% \begin{overpic}[width=0.495\textwidth]%
%%   {../../../lab/bench/fig3A/fig3A.pdf}
%%   \put(18.,66.5){\textbf{A}}
%% \end{overpic}
%% \begin{overpic}[width=0.495\textwidth]%
%%   {../../../lab/bench/fig3B/fig3B.pdf}
%%   \put(18.,66.5){\textbf{A}}
%% \end{overpic}
\includegraphics[width=\textwidth]{%
  /home/fh/sci/rsc/nrnd_pairs/pub/arxiv_16/figures/sym_fail/sym_fail_figure.png}
\caption{Relative overrepresentation of bidirectional connections $\varrho$ is sustained when connection probabilities are only approximately symmetric in pairs. \textbf{A} Illustration how the conditional density function $f_{P_{ji} | P_{ij}} (y\,\vert\, x)$ in \eqref{eq:fpijpji} transitions from equality to independence of the random variables $P_{ij}$ and $P_{ji}$ with increasing $\sigma$. Here we chose $f_{P_{ij}} (y) = f^T_{\alpha,\beta}(y)$ from $\eqref{eq:fTab}$, with $\alpha=0.248$ and $\beta$ chosen such that the overall connection probability $\mu=0.1$, and it is $P_{ij}$ fixed as $x=0.15$. For $\sigma=1$ the conditional density function becomes indistinguishable from  $f^T_{\alpha,\beta}(y)$ in the plot. \textbf{B} Relative occurrence of reciprocally connected pairs $\varrho$  } 
\label{fig:sym_fail}
\end{figure}


%
%% and
%% %
%% \begin{align}
%%   \lim_{\sigma \to \infty}   f_{P_{ji}|P_{ij}} (y \mid x) = f_{P_{ji}}(y)
%% \end{align}
%% Expression $\eqref{eq:rho}$ can then be written as
%% %
%% \begin{align}
%% \varrho = \frac{\E(P_{ij}P_{ji})}{{\E\left(P_{ij}\right)}^2} = \frac{\int_0^1 \int_0^1 xy\, f_{P_{ij}P_{ji}}(x,y) \, dx\, dy}{\left(\int_0^1 x f_{P_{ij}}(x)\, dx\right)^2}
%% \end{align}

