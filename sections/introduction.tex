%% Non-random network connectivity in a random graph can be modeled by assigning each connection a separate probability to exist. Some connections are more likely to be realized in the network, some less. In the limiting case a blue print. Here we assume a network with 

* The wiring of brain networks is important for functional processes (which, how)

* How does the brain establish these connections? Increasing evidence suggests that microcircuitry is highly non-random \cite{Song2005,Perin2011}. Thus not every connection is equally like established, rather some connections are more likely connected than others.

* A specific topic of interest is the overrepresentation of bidirectional connections \cite{Markram1997,Song2005}. Explanation of what overrepresentation means.

* Review of different studies that studied the emergence or functional implicatioins of recip connections (for example \cite{Clopath2010})

* In this theoretical work we study non-random network connectivity in a network model in which each connection has a separate probability to exist.

* In cortical circuits connection probabilities might be affected by many different parameters: interneuron distance, age, location,...

* If the processes in neural circuits that affect connection probability are symmetric, that is projection to a specific target is as likely as receiving input from this target, we show that bidirectional connections are necessarily occurring more often than in a random network as long as some connections are more likely than others.

* Why this is an important result: Overrepresentation of bidirectional connections maybe more a symptom of non-randomness + symmetry. Insight allows the formulation of possibly more important questions (weights of bidirectional connections etc..)

* A quantitative analysis of a discrete and continuous distribution helps put experimental results into context and can serve as reference for future studies



