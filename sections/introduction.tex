%% Non-random network connectivity in a random graph can be modeled by assigning each connection a separate probability to exist. Some connections are more likely to be realized in the network, some less. In the limiting case a blue print. Here we assume a network with 

* The wiring of brain networks is important for functional processes (which, how)

How does the brain establish these connections? Increasing evidence suggests that microcircuitry is highly non-random \cite{Song2005,Perin2011}. Such experimental studies suggest that not every connection is equally likely to be established, rather some pairs of neurons are more likely connected than others.

Here, the relative occurrence of bidirectionally connected pairs has been of particular interest \cc{[cite bidirectional review?]}. Comparing the amount of bidirectional connections as one would expect in a comparable random network with the number of reciprocally connected pairs identified through experiment, many studies using paired whole-cell recordings report a relative overrepresentation of such bidirectional pairs \cite{Markram1997,Song2005}.

$\cdots$

The prevalence of bidirectional connectivity has thus been established as an important measure for non-randomness of a network. However, the exact relationship between non-randomness and relative reciprocity has not been explained. Here, we show that . Conversely, an absence of bidirectional a


* Review of different studies that studied the emergence or functional implicatioins of recip connections (for example \cite{Clopath2010})

* In this theoretical work we study non-random network connectivity in a network model in which each connection has a separate probability to exist.

* In cortical circuits connection probabilities might be affected by many different parameters: interneuron distance, age, location,...

* If the processes in neural circuits that affect connection probability are symmetric, that is projection to a specific target is as likely as receiving input from this target, we show that bidirectional connections are necessarily occurring more often than in a random network as long as some connections are more likely than others.

* Why this is an important result: Overrepresentation of bidirectional connections maybe more a symptom of non-randomness + symmetry. Insight allows the formulation of possibly more important questions (weights of bidirectional connections etc..)

* A quantitative analysis of a discrete and continuous distribution helps put experimental results into context and can serve as reference for future studies



