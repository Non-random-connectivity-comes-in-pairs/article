%% Non-random network connectivity in a random graph can be modeled by assigning each connection a separate probability to exist. Some connections are more likely to be realized in the network, some less. In the limiting case a blue print. Here we assume a network with 

%% * The wiring of brain networks is important for functional processes (which, how)

%How does the brain establish these connections?
Increasing evidence shows that cortical microcircuitry is highly structured \cite{Song2005,Perin2011}. Not every connection is equally likely to be established, rather some pairs of neurons are more likely connected than others. In this context, the relative occurrence of bidirectionally connected pairs has been of particular interest. Using data obtained from paired whole-cell recordings in cortical slices, the amount of bidirectionally connected pairs was compared to the number of reciprocal pairs as one would expect in a random network with the same overall connection probability. Connectivity of layer 5 pyramidal neurons in the rat visual cortex \cite{Song2005} and somatosensory cortex \cite{Markram1997, Perin2011} was shown to have a significantly stronger reciprocity than expected.


%% Comparing the amount of reciprocally connected pairs as one would expect to find in an equivalent random network with the number of bidirectionally connected pairs identified by paired whole-cell recordings in cortical slices, a relative overrepresentation of such bidirectional pairs has been reported for layer 5 pyramidal neurons in the rat visual cortex \cite{Song2005} and layer 5 pyramidal cells in the rat somatosensory cortex \cite{Markram1997,Perin2011}.

The prevalence of bidirectional connectivity has since been established as an important measure for the non-randomness of a network \cite{Lefort2009}. However, the exact relationship between non-randomness and relative reciprocity has not been explained. Here, we model cortical circuitry as random networks in which each possible connection has a separate probability to exist. Using this model we're able to show that any non-random connectivity, expressed as higher connection probabilities in some edges and lower probabilities in others, necessarily induces a relative overrepresentation of bidirectional connections as long as connection probabilities remain symmetric within pairs. Quantitatively, we analyze reciprocity in networks with a discrete and a continuous distribution in connection probabilities to demonstrate that a relative occurrence of bidirectional connections as reported from experimental studies can be easily obtained in these models.


%% For two distributions of connection probabilities, the discrete two-point distribution and continuous gamma distribution, we quantify how a more organized structure increases reciprocity in the network.


%% * Review of different studies that studied the emergence or functional implicatioins of recip connections (for example \cite{Clopath2010})

%% * In this theoretical work we study non-random network connectivity in a network model in which each connection has a separate probability to exist.

%% * In cortical circuits connection probabilities might be affected by many different parameters: interneuron distance, age, location,...

%% * If the processes in neural circuits that affect connection probability are symmetric, that is projection to a specific target is as likely as receiving input from this target, we show that bidirectional connections are necessarily occurring more often than in a random network as long as some connections are more likely than others.

%% * A quantitative analysis of a discrete and continuous distribution helps put experimental results into context and can serve as reference for future studies



