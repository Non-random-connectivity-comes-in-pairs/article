
We're discussing now a network model in which node-to-node connection probabilities are randomly distributed. One might think for example of a network in which connection probability decreases with distance.

Remembering the notation from above, we write
\[
P(X_{ij}=1) = P_{ij}
\]
where $P=P_{ij}$ are identically distributed continuous random variables with values in $[0,1]$. We write the of probability density function of the $P_{ij}$ as $f_{P_{ij}}$.

First, the overall connection probability $\bar{p}$ in the network is just the expected value of the $P_{ij}$,
\[
\bar{p} = \E(P_{ij}) = \int_0^1 p f_{P_{ij}}(p)\, dp.
\]

Again, the \enquote{expected} probability to find a bidirectional connection would be given as
\[
\bar{p}^2 = \E(P_{ij})^2.
\]
However, the chance to have an bidirectional connection in a pair is the product of the two connection probabilities of the two nodes. In our model, these probabilities $P=P_{ij}$ and $Q=P_{ji}$ stem from a random distribution, given by their density functions $f_P = f_Q$.


\[
\int_0^1 \int_0^1 pq f_{PP}(p,q) \,dp\,dq
\]

If $P$ and $Q$ are independent


Otherwise, if the is symmetric in connection probabilities
