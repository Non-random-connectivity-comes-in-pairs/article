
We're discussing now a network model in which node to node connection probabilities are randomly distributed. One might think for example of a network in which connection probability decreases with distance.

Remembering the notation from above, we write
\[
P(X_{ij}=1) = P
\]
where P is a random variable with values in $[0,1]$.

First, the overall connection probability $\bar{p}$ in the network is just the expected value of $P$,
\[
\bar{p} = \E(P) = \int_0^1 p f_P(p)\, dp.
\]

Again, the \enquote{expected} probability to find a bidirectional connection would be given as $\bar{p}^2 = \E(P)^2$.

However, the chance to have an bidirectional connection in a pair is the product of the two connection probabilities of the two nodes. In our model, these probabilities $p$ and $q$ stem from a random distribution, given for example by its density function $f_P$.


\[
\int_0^1 \int_0^1 pq f_{PP}(p,q) \,dp\,dq
\]

If $P$ and $Q$ are independent


Otherwise, if the is symmetric in connection probabilities
